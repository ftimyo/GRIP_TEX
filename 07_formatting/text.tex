% text.tex - examples of various text formatting techniques

% Andrew Roberts - September 2003

\documentclass[english]{article}

\usepackage{times}
\usepackage{babel}
\usepackage{ulem}

\begin{document}

\normalem

\title{Latex Tutorial 7 (Formatting) Examples}
\author{Andrew Roberts}
\maketitle

\section{Text Formatting}

\subsection{Quotes}

\begin{tabular}{ l | l }

\verb|To `quote' in Latex.| & To `quote' in Latex. \\
\verb|To ``quote'' in Latex.| & To ``quote'' in Latex. \\
\verb|To ``quote" in Latex.| & To ``quote" in Latex. \\
\end{tabular}

\subsection{Dots and Dashes}

\begin{tabular}{l | l}
\verb|Ellipsis...| & Ellipsis... \\
\verb|Ellipsis \ldots| & Ellipsis \ldots \\
\end{tabular}

\vspace{\baselineskip}

\begin{tabular}{| l | l | l |}
\hline
Input & Output & Purpose \\ \hline
\verb|-| & - & inter-word \\
\verb|--| & -- & page range, 1--10 \\
\verb|---| & --- & punctuation dash --- like this \\
\hline

\end{tabular}

\subsection{Accents}

\begin{tabular}{| l l | l l |}
\hline
Command & Output & Command & Output \\
\hline
\verb|\`{o}| & \`{o} &
\verb|\'{o}| & \'{o} \\
\verb|\"{o}| & \"{o} &
\verb|\H{o}| & \H{o} \\
\verb|\^{o}| & \^{o} &
\verb|\~{o}| & \~{o} \\
\verb|\v{o}| & \u{o} &
\verb|\={o}| & \={o} \\
\verb|\b{o}| & \b{o} &
\verb|\.{o}| & \.{o} \\
\verb|\d{o}| & \d{o} &
\verb|\c{o}| & \c{o} \\
\verb|\r{o}| & \r{o} &
\verb|\t{oo}| & \t{oo} \\
\verb|\i| & \i & &\\
\hline
\end{tabular}

\subsection{Symbols}

\begin{tabular}{|l c | l c |}
\hline
Command & Symbol & Command & Symbol \\
\hline
\verb|\%| & \% &
\verb|\#| & \# \\
\verb|\$| & \$ &
\verb|\&| & \& \\
\verb|\{| & \{ &
\verb|\}| & \} \\
\verb|\_| & \_ &
\verb|\S| & \S \\
\verb|\P| & \P &
\verb|\dag| & \dag \\
\verb|\ddag| & \ddag &
\verb|\textbackslash| & \textbackslash \\
\verb|\textbar| & \textbar &
\verb|\textless| & \textless \\
\verb|\textgreater| & \textgreater &
\verb|\textemdash| & \textemdash \\
\verb|\textendash| & \textendash &
\verb|\textregistered| & \textregistered \\
\verb|\texttrademark| & \texttrademark &
\verb|\textquestiondown| & \textquestiondown \\
\verb|\textexclamdown| & \textexclamdown &
\verb|\textcircled{a}| & \textcircled{a}\\
\verb|\textsuperscript{a}| & \textsuperscript{a} &
\verb|\copyright| & \copyright \\
\verb|\pounds| & \pounds & & \\
\hline
\end{tabular}

\subsection{Emphasizing Text}

I want to \emph{emphasize} a word.

\subsection{Font styles}

\begin{tabular}{| l | l |}
\hline
Command & Style \\
\hline
\verb|\textnormal{...}| & \textnormal{document font family} \\ \hline
\verb|\emph{...}| & \emph{emphasis} \\ \hline
\verb|\textrm{...}| & \textrm{roman font family} \\ \hline
\verb|\textsf{...}| & \textsf{sans serif font family} \\ \hline
\verb|\texttt{...}| & \texttt{typewriter font family} \\ \hline
\verb|\textup{...}| & \textup{upright shape} \\ \hline
\verb|\textit{...}| & \textit{italic shape} \\ \hline
\verb|\textsl{...}| & \textsl{slanted shape} \\ \hline
\verb|\textsc{...}| & \textsc{small capitals} \\ \hline
\verb|\textbf{...}| & \textbf{bold} \\ \hline
\verb|\textmd{...}| & \textmd{normal weight and width} \\
\hline
\end{tabular}

\vspace{\baselineskip}

Using the \texttt{ulem} package, it is possible to \uline{underline}
words, add \uwave{under-wave} and \sout{strike out}.

\subsection{Font sizes}

\begin{tabular}{| l | c |}
\hline
Command & Output \\
\hline
\verb|\tiny| & \tiny{sample text} \\
\verb|\scriptsize| & \scriptsize{sample text} \\
\verb|\footnotesize| & \footnotesize{sample text} \\
\verb|\small| & \small{sample text} \\
\verb|\normalsize| & \normalsize{sample text} \\
\verb|\large| & \large{sample text} \\
\verb|\Large| & \Large{sample text} \\
\verb|\LARGE| & \LARGE{sample text} \\
\verb|\huge| & \huge{sample text} \\
\verb|\Huge| & \Huge{sample text} \\
\hline
\end{tabular}

\end{document}
