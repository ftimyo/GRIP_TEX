% specifier.tex - examples on using the @{...} construct in the tabular
%                 environment.

% Andrew Roberts - September 2003

\documentclass[english]{article}

\usepackage{times}
\usepackage{babel}

\begin{document}

\title{Latex Tutorial 4 (Tables) Examples \\
       @-expressions}

\author{Andrew Roberts}
\maketitle


A common use of the @ specifier is to align scientific tables on the
decimal place.

\vspace{\baselineskip}

\begin{tabular}{r@{.}l}
3&14159\\
16&2\\
123&456\\
\end{tabular}

\vspace{\baselineskip}

Examples for altering horizontal spacing:

\vspace{\baselineskip}

\begin{tabular}{| l | l |}
\hline
stuff & stuff \\ \hline
stuff & stuff \\
\hline
\end{tabular}

\vspace{\baselineskip}

\begin{tabular}{|@{}l|l@{}|}
\hline
stuff & stuff \\ \hline
stuff & stuff \\
\hline
\end{tabular}

\vspace{\baselineskip}

\begin{tabular}{|@{}l@{}|l@{}|}
\hline
stuff & stuff \\ \hline
stuff & stuff \\
\hline
\end{tabular}

\vspace{\baselineskip}

\begin{tabular}{|@{}l@{}|@{}l@{}|}
\hline
stuff & stuff \\ \hline
stuff & stuff \\
\hline
\end{tabular}

\end{document}
