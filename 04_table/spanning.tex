% spanning.tex - How to use the multirow and multicolumn commands to
%                span rows and columns.

% Andrew Roberts - September 2003

\documentclass[english]{article}

\usepackage{times}
%\usepackage{babel}
\usepackage{multirow}

\begin{document}

\title{Latex Tutorial 4 (Tables) Examples\\
       Spanning rows and columns}

\author{Andrew Roberts}
\maketitle

An example of \texttt{\textbackslash multicolumn}.

\vspace{\baselineskip}

\begin{tabular}{|l|l|}
\hline
\multicolumn{2}{|c|}{Team sheet} \\
\hline
GK & Paul Robinson \\
LB & Lucus Radebe \\
DC & Michael Duberry \\
DC & Dominic Matteo \\
RB & Didier Domi \\
MC & David Batty \\
MC & Eirik Bakke \\
MC & Jody Morris \\
FW & Jamie McMaster \\
ST & Alan Smith \\
ST & Mark Viduka \\
\hline
\end{tabular}

\vspace{\baselineskip}

An example of \texttt{\textbackslash multirow}.

\vspace{\baselineskip}

\begin{tabular}{|l|l|l|}
\hline
\multicolumn{3}{|c|}{Team sheet} \\
\hline
Goalkeeper & GK & Paul Robinson \\ \hline
\multirow{4}{*}{Defenders} & LB & Lucus Radebe \\
& DC & Michael Duberry \\
& DC & Dominic Matteo \\
& RB & Didier Domi \\ \hline
\multirow{3}{*}{Midfielders} & MC & David Batty \\
 & MC & Eirik Bakke \\
 & MC & Jody Morris \\ \hline
Forward & FW & Jamie McMaster \\ \hline
\multirow{2}{*}{Strikers} & ST & Alan Smith \\
 & ST & Mark Viduka \\
\hline
\end{tabular}
\end{document}
