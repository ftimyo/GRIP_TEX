% tabular_star.tex - examples of using the tabular* environment

% Andrew Roberts - September 2003

\documentclass[english]{article}

\usepackage{times}
\usepackage{babel}

\begin{document}

\title{Latex Tutorial 4 (Tables) Examples \\ Using tabular*}
\author{Andrew Roberts}
\maketitle

Tabular* requires an extra argument before the column description for
the user to supply the total width of the table. The following table has
had its width set to be 3/4 of text width (i.e., page width - left
margin - right margin).

\vspace{\baselineskip}

\begin{tabular*}{0.75\textwidth}{ | c | c | c | r | }
     %{@{\extracolsep{\fill}}cccr}
  \hline
  label 1 & label 2 & label 3 & label 4 \\
  \hline  % put a line under headers
  item 1  & item 2  & item 3  & item 4  \\
  \hline
\end{tabular*}

\vspace{\baselineskip}

However, that doesn't look quite as intended. The columns are still at
their natural width (just wide enough to fit their contents whilst the
rows are as wide as the table width specified.  This looks very ugly.
The reason for the mess is that you must also explicitly insert extra
column space.

\vspace{\baselineskip}

\begin{tabular*}{0.75\textwidth}{@{\extracolsep{\fill}} | c | c | c | r | }
  \hline
  label 1 & label 2 & label 3 & label 4 \\
  \hline
  item 1  & item 2  & item 3  & item 4  \\
  \hline
\end{tabular*}

\end{document}
