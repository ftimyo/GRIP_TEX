% wrapping.tex - Illustrating the wrapfig package for wrapping text
%                around figures

% Andrew Roberts - September 2003

\documentclass[english]{article}

\usepackage{times}
\usepackage{babel}
\usepackage{graphicx}
\usepackage{wrapfig}

\begin{document}

\title{Latex Tutorial 6 (Floats, Figures and Captions) Examples
\thanks{Information about the Toucan was provided by Microsoft Encarta
Online Encyclopedia 2003 (\texttt{http://encarta.msn.com})}}

\author{Andrew Roberts}
\maketitle

\section*{Toucan}

\begin{wrapfigure}{r}{40mm}
  \begin{center}
    \includegraphics{toucan.eps}
  \end{center}
  \caption{The Toucan}
\end{wrapfigure}

Toucan, common name for members of the bird family distinguished by
colorful, enormous but lightweight beaks, that inhabit tropical America.
The family includes six genera and about 40 species.


Toucans range in size from 18 to 63 cm (from 7 to 25 in). The body is
short and thick; the tail is rounded, varying in length in the different
species from half the length to almost the whole length of the body. The
neck is short and thick. At the base of the full width and depth of the
head is a huge, brightly colored beak, measuring in some of the larger
species more than half the length of the body. The tongue of the toucan
is long, narrow, and singularly frayed on each side, possibly to add to
its sensibility as an organ of taste. The legs are strong and rather
short, with large scales. The toes are arranged in pairs, with the first
and fourth turned backward. Males and females are alike in color. The
plumage in the genus containing the largest toucans is generally black,
with touches of white, yellow, and scarlet. In the smaller aracari
toucans, the underparts are yellow, crossed by one or more black or red
bands, and the edges of the upper half of the beak are prominently
saw-toothed. The toucanets are mostly green, with blue markings.

Toucans usually live in pairs or small flocks. They feed chiefly on
fruit, and can manipulate small berries at the tip of the bill with
great dexterity. They also eat small birds and lizards. They lay white,
glossy eggs in hollows of trees, making little if any nest for them. The
young are hatched completely naked, without any down.

Scientific classification: Toucans make up the family
\emph{Ramphastidae}. The largest toucans are classified in the genus
\emph{Ramphastos}, the aracari toucans in the genus \emph{Pteroglossus},
and the toucanets in the genus \emph{Aulacorhynchus}.

\end{document}
